%%
%% This document is released under the Creative Commons Attribution-Share Alike 3.0 Licence. The full
%% text of this licence can be found at http://creativecommons.org/licenses/by-sa/3.0/.
%%
\documentclass[xcolor=pdftex,dvipsnames]{beamer}

\usepackage{ucs}
\usepackage[utf8x]{inputenc}
\usepackage[english]{babel}
\usepackage{hyperref}
\usepackage{pgf}
\usepackage{tikz}
\usepackage{alltt}
\usepackage{bold-extra}

\usetikzlibrary{arrows}

\hypersetup{
    colorlinks=false,
    pdftitle={Git Tutorial part I: An Introduction to Source Code Management Systems},
    pdfauthor={Jesper Louis Andersen and Alexander Færøy},
    pdflang={en},
    pdfcreator={pdfLaTeX and hyperref},
    pdfproducer={pdfLaTeX and hyperref}
}

\setbeamercolor{structure}{fg=Green!75!black}
\setbeamertemplate{navigation symbols}{}
\setbeamertemplate{footline}{}

\title{Git}
\subtitle{Source code management the UNIX way}

\author{\href{mailto:jesper.louis.andersen@gmail.com}{Jesper Louis Andersen}
        \and
        \href{mailto:ahf@0x90.dk}{Alexander Færøy}}

\newenvironment{usercmd}[1]{\begin{alltt}\$ \textbf{#1}}{\end{alltt}}

\begin{document}

\frame{\titlepage}

\begin{frame}{Table of Contents}
    \tableofcontents
\end{frame}

\section{Introduction}
\begin{frame}{The History of Git}
    \begin{itemize}
        \item Initially designed by Linus Torvalds.
        \item Git became self-hosted on April 7, 2005.
        \item The Linux Kernel project moved to Git 9 days later.
        \item Today, thousands of projects
    \end{itemize}
\end{frame}
\begin{frame}{What is?}
  \begin{itemize}
  \item Storage system (Persistence)
  \item Revision control on top
  \item UNIX philosophy in the tool-set
  \item Key advantage: Flexibility
  \item Key disadvantage: Relatively steep learning curve
  \end{itemize}
\end{frame}
\begin{frame}
  \frametitle{What can it do?}
  \begin{itemize}
  \item Projects \emph{evolve}.
  \item Git, like SVN, Darcs, CVS, Hg and Bazaar, manages project
    data, tracks history, facilitates collaboration etc.
  \end{itemize}
\end{frame}

\section{Storage Model}
\begin{frame}{Concept: Persistence}
  \begin{itemize}
  \item Databases
  \item Functional languages
  \item Accounting/Finance
  \item \emph{Never} overwrite old data
  \end{itemize}
\end{frame}
\begin{frame}{Term: Blob}
  \begin{itemize}
  \item A Blob stores \emph{content}, ie what is in a file
  \item Compressed, for saving storage and disk reads
  \item Identified by an SHA1 checksum
  \item Note that SHA1 is 2nd preimage resistant (still)
  \end{itemize}
\end{frame}
\begin{frame}{Term: Tree}
  \begin{itemize}
  \item A Tree contains a list of references paired with meta-data
  \item References points to either blobs or other trees
  \item This is used to map (among other things) the directory
    structure
  \item Identified by an SHA1 checksum
  \end{itemize}
\end{frame}
\begin{frame}{Term: Commit}
  \begin{itemize}
  \item A Commit references a Tree and some parent commits
  \item Identified by an SHA1 checksum
  \item Consistency: Use the SHA1-sums
  \end{itemize}
\end{frame}
\begin{frame}{Tag}\end{frame}
\begin{frame}{Storage}
  \begin{itemize}
  \item Reuse existing subtrees (dedup)
  \item Always write a new blob
  \end{itemize}
\end{frame}
\begin{frame}{Packs}
  \begin{itemize}
  \item Heuristic
  \item fsync()
  \item GC
  \end{itemize}
\end{frame}
\begin{frame}{Index}\end{frame}
\begin{frame}{Index(2)}\end{frame}
\begin{frame}{Distribution(1)}\end{frame}
\begin{frame}{Distribution(2)}\end{frame}

\section{Practical git}

\section{Question Time}
\begin{frame}{Question Time}
    \begin{itemize}
        \item Questions from the audience?
        \item \url{http://github.com/ahf/sslug-workshop-git/}
    \end{itemize}
\end{frame}

\end{document}

% vim: set et spell spelllang=en tw=120 :

%%% Local Variables: 
%%% mode: latex
%%% TeX-master: t
%%% End: 
