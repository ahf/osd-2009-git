%%
%% This document is released under the Creative Commons Attribution-Share Alike 3.0 Licence. The full
%% text of this licence can be found at http://creativecommons.org/licenses/by-sa/3.0/.
%%
\documentclass[xcolor=pdftex,dvipsnames]{beamer}

\usepackage{ucs}
\usepackage[utf8x]{inputenc}
\usepackage[english]{babel}
\usepackage{hyperref}
\usepackage{pgf}
\usepackage{tikz}
\usepackage{alltt}
\usepackage{bold-extra}
\usepackage{verbatim}

\usetikzlibrary{arrows}

\hypersetup{
    colorlinks=false,
    pdftitle={Git Tutorial part I: An Introduction to Source Code Management Systems},
    pdfauthor={Jesper Louis Andersen and Alexander Færøy},
    pdflang={en},
    pdfcreator={pdfLaTeX and hyperref},
    pdfproducer={pdfLaTeX and hyperref}
}

\setbeamercolor{structure}{fg=Green!75!black}
\setbeamertemplate{navigation symbols}{}
\setbeamertemplate{footline}{}

\title{Git}
\subtitle{Source code management the UNIX way}

\author{\href{mailto:jesper.louis.andersen@gmail.com}{Jesper Louis Andersen}
        \and
        \href{mailto:ahf@0x90.dk}{Alexander Færøy}}

\newenvironment{usercmd}[1]{\begin{alltt}\$ \textbf{#1}}{\end{alltt}}

\begin{document}

\frame{\titlepage}

\begin{frame}{Table of Contents}
    \tableofcontents
\end{frame}

\section{Introduction}
\begin{frame}{The History of Git}
    \begin{itemize}
        \item Initially designed by Linus Torvalds.
        \item Git became self-hosted on April 7, 2005.
        \item The Linux Kernel project moved to Git 9 days later.
        \item Today, thousands of projects
    \end{itemize}
\end{frame}
\begin{frame}{What is?}
  \begin{itemize}
  \item Storage system (Persistence)
  \item Revision control on top
  \item UNIX philosophy in the tool-set
  \item Key advantage: Flexibility
  \item Key disadvantage: Relatively steep learning curve
  \end{itemize}
\end{frame}
\begin{frame}
  \frametitle{What can it do?}
  \begin{itemize}
  \item Projects \emph{evolve}.
  \item Git, like SVN, Darcs, CVS, Hg and Bazaar, manages project
    data, tracks history, facilitates collaboration etc.
  \end{itemize}
\end{frame}

\section{Storage Model}
\begin{frame}{Concept: Persistence}
  \begin{itemize}
  \item Databases
  \item Functional languages
  \item Accounting/Finance
  \item \emph{Never} overwrite old data
  \item Git breaks the rules in a few places
  \end{itemize}
\end{frame}
\begin{frame}{Term: Blob}
  \begin{itemize}
  \item A Blob stores \emph{content}, ie what is in a file
  \item Compressed, for saving storage and disk reads
  \item Identified by an SHA1 checksum
  \item Note that SHA1 is 2nd preimage resistant (still)
  \end{itemize}
\end{frame}
\begin{frame}{Term: Tree}
  \begin{itemize}
  \item A Tree contains a list of references paired with meta-data
  \item References points to either blobs or other trees
  \item This is used to map (among other things) the directory
    structure
  \item Identified by an SHA1 checksum
  \end{itemize}
\end{frame}
\begin{frame}{Term: Commit}
  \begin{itemize}
  \item A Commit references a Tree and some parent commits
  \item Identified by an SHA1 checksum
  \item Consistency: Use the SHA1-sums
  \end{itemize}
\end{frame}
\begin{frame}{Tag}
  \begin{itemize}
  \item A Tag is a reference to a Commit
  \item These beasts can be cryptographically signed
  \end{itemize}
\end{frame}
\begin{frame}{Concept: Storage principles}
  \begin{itemize}
  \item Reuse existing subtrees (dedup)
  \item Always write a new object
  \end{itemize}
\end{frame}
\begin{frame}{Concept: Packs}
  \begin{itemize}
  \item A pack compresses a set of objects into a single file
  \item Delta Compression: Order objects by (type, basename,
    desc. size) lexicographically.
  \item Delta Compression: Run a sliding window on the order, search
    for delta-coding opportunities.
  \item Storage: Recency ordered in the pack from the HEAD
  \item Storage: Good locality
  \item Only ``destructive operation'' -- fsync()
  \item Ideas the same as garbage collection (command is \texttt{git gc})
  \end{itemize}
\end{frame}
\begin{frame}{Concept: Index}
Conceptual Index location:
$$
  WorkingDir \leftrightarrow Index \leftrightarrow Storage
$$
\end{frame}
\begin{frame}{Concept: Index(2)}
  \begin{itemize}
  \item Index is \emph{mutable}, storage is \emph{immutable}.
  \item UI View: Staging area for the next commit.
  \item Backend View: Directory cache for speeding up operations.
  \item Git owes much of its speed to the index.
  \end{itemize}
\end{frame}
\begin{frame}{Concept: Distribution(1)}
  \begin{itemize}
  \item A \emph{Ref} is a stick-it-note we can place on a commit.
  \item Refs are used for \emph{branches}, different paths in development.
  \item Refs are used for certain magic markers in the storage tree.
  \item Some refs are \emph{local}, other are \emph{remote}
  \end{itemize}
\end{frame}
\begin{frame}{Concept: Distribution(2)}
  \begin{itemize}
  \item Distribution happens by cloning the repository -- copying all
    its contents
  \item The storage model makes this approach feasible
  \item Fetches: Copy the difference, track where the refs moved to.
  \item Pushes:  Copy the difference, track where the refs moved to.
  \end{itemize}
\end{frame}

\section{Practical git}
\begin{frame}
  \frametitle{How to start}
  \begin{itemize}
  \item Grab a tutorial
  \item We think you gained a lot with the knowledge about the
    storage.
  \end{itemize}
\end{frame}

\begin{frame}
  \frametitle{git add}
  \begin{itemize}
  \item When you run \texttt{git add}, you add things to the index
  \item \texttt{git commit} snapshots what is in the index alone.
  \end{itemize}
\end{frame}

\begin{frame}[fragile]
  \frametitle{merge/rebase}
  There is a difference between merge and rebase. Usually you want to
  rebase in git:
\begin{verbatim}
      A---B---C topic
     /
D---E---F---G master
\end{verbatim}
\begin{verbatim}
              A--B--C topic
             /
D---E---F---G master
 \end{verbatim}
\end{frame}

\begin{frame}
  \frametitle{merge/rebase (2)}
  \begin{itemize}
  \item Note: Patches already in upstream are skipped
  \item Keeps a linear development path -- easier to track for other
    people
  \item As long as history is \emph{local} it can be rewritten!
    (Persistence break!)
  \item Do not rewrite public branches, or people can't track it
  \end{itemize}
\end{frame}

\begin{frame}
  \frametitle{Moving files}
  \begin{itemize}
  \item Controversy: git does not track that a file has been moved. It
    tracks it as a content change.
  \item A heuristic decides if the file is a move when browsing.
  \item Usually gets it right in most cases
  \end{itemize}
\end{frame}

\begin{frame}
  \frametitle{Publishing your changes}
  \begin{itemize}
  \item One simple way is \texttt{github.com} or \texttt{gitorius.org}
  \item Provides an outlet for publishing your tree
  \item For internal hosting at companies, either pay Github or
  \item Use a web-interface on top of a repository, or
  \item Use no central repository system at all!
  \end{itemize}
\end{frame}

\begin{frame}
  \frametitle{Editor integration etc.}
  \begin{itemize}
  \item Emacs: magit-mode
  \item Vim: There is git-vim, or just use the command line
  \item gitk -- interface to git via Tk
  \end{itemize}
\end{frame}


\section{Question Time}
\begin{frame}{Question Time}
    \begin{itemize}
        \item Questions from the audience?
        \item \url{http://github.com/ahf/sslug-workshop-git/}
    \end{itemize}
\end{frame}

\end{document}

% vim: set et spell spelllang=en tw=120 :

%%% Local Variables: 
%%% mode: latex
%%% TeX-master: t
%%% End: 
